% Handouts
%
% Tema 3. Sistemes Lineals. Mètodes Iteratius 
%

\documentclass[slidetop,compress,10pt]{beamer}

\mode<presentation>{\usetheme{SimplePlus}}
\setbeamertemplate{theorems}[numbered]
\setbeamertemplate{caption}[numbered]
\setbeamertemplate{itemize item}[triangle]
\setbeamertemplate{itemize subitem}[circle]
\setbeamertemplate{frametitle continuation}{%
\usebeamerfont{frametitle}
(\insertcontinuationcountroman)
}

%\apptocmd{\frame}{}{\justifying}{} % Allow optional arguments after frame.

\usepackage{mathptmx}
\usepackage[scaled=0.9]{helvet}
\usepackage{courier}

\usepackage{ragged2e}
\usepackage{amsmath,amssymb}
\usepackage{amsfonts}
\usepackage{mathtools}
\usepackage{wrapfig}
\usepackage{cancel}
\usepackage{fontawesome}
\usepackage{caption}
\usepackage{color}

\usepackage[utf8]{inputenc}
\usepackage[english]{babel}
\selectlanguage{english}
\uselanguage{English}
\languagepath{English}

\usepackage{ifpdf}
\usepackage{numprint}
\usepackage{oldstyle}
\graphicspath{ {_figures/} }
\DeclareGraphicsExtensions{.png}
\usepackage{subfigure}

\usepackage{yhmath}

\usepackage[round]{natbib}

\hypersetup{
  pdfauthor   = {J.R. Pacha},
  pdftitle    = {FEM: Weak formulation},
  pdfsubject  = {FEM},
  pdfpagemode = {FullScreen},
  pdfstartview = {},
  colorlinks  = {true},
%  bookmarks   = {true},
%  pagebackref = {true},
  bookmarksnumbered = {true},
  hyperindex  = {true}
}
% \pdfadjustspacing=1

\usepackage{url}
%\setcounter{section}{3}

% THEOREMS -------------------------------------------------------
\newtheorem{thm}{Teorema}%[section]
\newtheorem{cor}[thm]{Corollary}
\newtheorem{lem}[thm]{Lemma}
\newtheorem{prop}{Proposition}%[thm]{Proposició}
\theoremstyle{definition}
\newtheorem{defn}{Definiton}%[thm]{Definici\'{o}}
\theoremstyle{remark}
%\newtheorem{rem}[thm]{Remarca}
\newtheorem{rem}{Remark}
%\newtheorem{ex}[thm]{Exemple}
\newtheorem{ex}{Example}
%\numberwithin{equation}{section}
\newtheorem{exc}{Exercise}
\newtheorem{demo}{Proof} 
\newtheorem*{demo*}{Proof} 

% Algorismes -----------------------------------------------------
\usepackage[noline,plain]{algorithm2e}
\setlength{\AlCapSkip}{0.7em}

\usepackage{hyperref}

% MATH -----------------------------------------------------------
\newcommand{\norm}[1]{\left\Vert#1\right\Vert}
\newcommand{\abs}[1]{\left\vert#1\right\vert}
\newcommand{\set}[1]{\left\{#1\right\}}
\newcommand{\bs}[1]{\ensuremath{\boldsymbol{#1}}}
\newcommand{\C}{\ensuremath{\mathbb{C}}}
\newcommand{\R}{\ensuremath{\mathbb{R}}}
\newcommand{\N}{\ensuremath{\mathbb{N}}}
\newcommand{\M}{\ensuremath{\mathbb{M}}}
\newcommand{\Z}{\ensuremath{\mathbb{Z}}}
\newcommand{\K}{\ensuremath{\mathbb{K}}}
\newcommand{\eps}{\ensuremath{\epsilon}}
\newcommand{\veps}{\ensuremath{\varepsilon}}
\newcommand{\To}{\longrightarrow}
\newcommand{\BX}{\mathbf{B}(X)}
\newcommand{\A}{\mathcal{A}}
\newcommand{\Or}{\mathcal{O}}
\newcommand{\Id}[1]{Id_{#1}}
\newcommand{\argcosh}{\ensuremath{\mathrm{argcosh}}\,}
\newcommand{\argsinh}{\ensuremath{\mathrm{argsinh}}\,}
\newcommand{\argtanh}{\ensuremath{\mathrm{argtanh}}\,}
\newcommand{\argcoth}{\ensuremath{\mathrm{argcoth}}\,}
\newcommand{\rme}{\mathrm{e}}
\newcommand{\rmi}{\mathrm{i}}
\newcommand{\rmd}{\mathrm{d}}
\newcommand{\re}{\mathrm{Re\:}}
\newcommand{\im}{\mathrm{Im\:}}
\newcommand{\wt}[1]{\ensuremath{\widetilde{#1}}}
\DeclareMathOperator{\Hessian}{Hess}
\DeclareMathOperator*{\maxc}{\text{\rm m\`{a}x}}
\DeclareMathOperator*{\limc}{\text{\rm l\'{\i}m}}
\DeclareMathOperator*{\minc}{\text{\rm m\'{\i}n}}
\newcommand{\interior}[1]{\ensuremath{\mathring{#1}}}
\DeclareMathOperator{\ext}{Ext}
\DeclareMathOperator{\fr}{Fr}
\DeclareMathOperator{\Dom}{\mathcal{D}}
\newcommand{\dom}[1]{\Dom\left(#1\right)}
\DeclareMathOperator{\Rang}{\mathcal{R}}
\newcommand{\rang}[1]{\Rang\left(#1\right)}
\newcommand{\Mat}[2]{\ensuremath M_{#1}(#2)}
\newcommand{\bracket}[2]{\ensuremath\left\langle #1, #2\right\rangle}
\newcommand{\verteq}{\rotatebox{90}{$\scriptstyle \,=$}}
\newcommand{\eqover}[1]{\substack{\verteq \\ \scriptstyle #1}}
\newcommand{\equnder}[1]{\substack{\scriptstyle #1 \\ \verteq}}

% ----------------------------------------------------------------
\newcommand{\gl}{\guillemotleft}
\newcommand{\gr}{\guillemotright}
\newcommand{\sst}{\ensuremath{\scriptstyle}}
\newcommand{\npos}[2]{\ensuremath{\oldstyle{\numprint[#1]{#2}}}}
\renewcommand{\thefootnote}{\fnsymbol{footnote}} %{$\scriptstyle (*)$}

% Counters -------------------------------------------------------
\newcounter{saveenum}
\beamersetuncovermixins{\opaqueness<1>{25}}{\opaqueness<2->{15}}
\setbeamertemplate{caption}[numbered]
\setbeamertemplate{enumerate item}{\arabic{enumi})}%
\setbeamertemplate{enumerate label}{\roman{enumi})}
\setbeamerfont{caption}{size=\scriptsize}

\newdimen\oldparindent    
\oldparindent=\parindent  %Keep the original parindent

\title[]%
{{\Large Mètodes Numèrics (240032)} \\
 \large Plane elasticity. Weak Formulation}

\date{\today}

%%----------------------------------------------------------------------------
\begin{document}
\frame{\titlepage}
\frame{\tableofcontents}

%%----------------------------------------------------------------------------
\begin{frame}[t,allowframebreaks]{Model equation for plane elasticity}%
  %\begin{defn}[

   \citep[In][chap.~11]{Reddy2006}
    for the plane elasticity problems, the the normal stress in the $x$
    direction, $\sigma_{xx} = \sigma_{xx}(x,y)$, the normal stress in the
    $y$ direction, $\sigma_{yy} = \sigma_{yy}(x,y)$, and the shear stress,
    $\sigma_{xy} = \sigma_{xy}(x,y)$, satisfy the BVP given by the two
    coupled system of PDE,

    \begin{equation}
      \left.
      \begin{aligned}
         \frac{\partial \sigma_{xx}}{\partial x} + \frac{\partial
         \sigma_{xy}}{\partial y} + f_{x}(x,y) &= 0\\
         \frac{\partial \sigma_{xy}}{\partial x} + \frac{\partial
        \sigma_{yy}}{\partial y} + f_{y}(x,y) &= 0
      \end{aligned}
      \right\}\quad\text{on $\Omega\subset \R^{2}$,}
      \label{eq:BVP-EDP}
    \end{equation}

    the \emph{natural} B.C. 
    \begin{equation}
      \left.
      \begin{aligned}
        t_{x}\equiv \sigma_{xx} n_{x} + \sigma_{xy} n_{y} = \hat{t}_{x}\\
        t_{y}\equiv \sigma_{xy} n_{x} + \sigma_{yy} n_{y} = \hat{t}_{y}
      \end{aligned}
      \right\}\quad \text{on $\Gamma_{\sigma}$,}
     \label{eq:BVP-NBC}
    \end{equation}

  and the \emph{essential} B.C.

  \begin{equation}
    u = \hat{u},\quad v = \hat{v}\quad\text{on $\Gamma_{u}$}
    \label{eq:BVP-EBC}
  \end{equation}

  where

  \begin{description}

    \item[$f_{x}$, $f_{y}$] are the components of the body force vector
      (per unit volume) along the $x$ and the $y$ direction respectively.

    \item[$n_{x}$, $n_{y}$] denote the components (or the direction
      cosines) of the unit normal vector on the boundary of $\Gamma$.

    \item[$\Gamma_{\sigma}$, $\Gamma_{u}$] are two disjoint pieces of
      $\Gamma$.

    \item[$\hat{t}_{x}$, $\hat{t}_{y}$] denote the components of the
      specified traction vector.

    \item[$\hat{u}$, $\hat{v}$] are the components of the specified
      displacement vector.

  \end{description}

  Only one element of each pair, $(u, t_{x})$ and $(v, t_{y})$, may be
  specified at a boundary point.

\end{frame}

%%----------------------------------------------------------------------------
\begin{frame}[t]{Relations Strain-displacement}\justifying

   On the one hand the components of the strain tensor 
   \begin{displaymath}
     \epsilon_{xx} = \epsilon_{xx}(x,y),\quad
     \epsilon_{xy} = \epsilon_{xy}(x,y),\quad
     \epsilon_{yy} = \epsilon_{yy}(x,y),
   \end{displaymath}

   are given by the components of the displacement field $u = u(x,y)$,
   $v = v(x,y)$ through the relations 

   \begin{displaymath}
     \epsilon_{xx} = \frac{\partial u}{\partial x},\quad
     2\epsilon_{xy} = \frac{\partial u}{\partial y} + \frac{\partial
     v}{\partial x},\quad
     \epsilon_{yy} = \frac{\partial v}{\partial y},
   \end{displaymath}

   or, in vector form,

   \begin{equation}
     %\renewcommand{\arraystretch}{1.5}
     \mathbf{\epsilon} \equiv 
     \begin{pmatrix}
       \epsilon_{xx}\\
       \epsilon_{yy}\\
       2\epsilon_{xy}
     \end{pmatrix} = 
     \begin{pmatrix}
       \frac{\partial u}{\partial{x}}\\[4pt]
       \frac{\partial v}{\partial{y}}\\[4pt]
       \frac{\partial u}{\partial y} + \frac{\partial v}{\partial x}
     \end{pmatrix}
   \end{equation}

\end{frame}

%%----------------------------------------------------------------------------
\begin{frame}[t]{Stress-Strain equation}\justifying
  On the other hand, the stress tensor is related by the strain tensor
  through the Stress-Strain equation
  % Put here the content of the 2nd slide

\end{frame}


%%----------------------------------------------------------------------------
\begin{frame}[t]{}\justifying
  % Put here the content of the 2nd slide

\end{frame}

%%----------------------------------------------------------------------------
\begin{frame}[t]\justifying
  % Put here the content of the 3rd slide
\end{frame}

%%-----------------------------------------------------------------------------
\begin{frame}[t]\justifying
  % Put here the content of the 4th slide

\end{frame}

%%-----------------------------------------------------------------------------
\begin{frame}[t]{Example: an algorithm with caption}
%% This is needed if you want to add comments in
%% your algorithm with \Comment
\DontPrintSemicolon

\SetKwComment{Comment}{\# }{}
\LinesNumbered
\centering
\scalebox{.8}{%
\begin{algorithm}[H]
  \captionof{algocf}{An algorithm with caption}\label{alg:two}
\KwData{$n \geq 0$}
\KwResult{$y = x^n$}
$y \gets 1$\;
$X \gets x$\;
$N \gets n$\;
\While{$N \neq 0$}{
  \eIf{$N$ is even}{
    $X \gets X \times X$\;
    $N \gets \frac{N}{2} $ \Comment*[r]{This is a comment}
  }{\If{$N$ is odd}{
      $y \gets y \times X$\;
      $N \gets N - 1$\;
    }
  }
}
\end{algorithm}
}
\end{frame}

%%----------------------------------------------------------------------------
\begin{frame}[allowframebreaks,t]%\justifying

  \frametitle{References}
%\bibliographystyle{alpha}

\bibliographystyle{plainnat}

\bibliography{biblio.bib}

%\nocite{*}

\end{frame}

\end{document}

%%% Local Variables:
%%% mode: latex
%%% TeX-master: t
%%% End:
